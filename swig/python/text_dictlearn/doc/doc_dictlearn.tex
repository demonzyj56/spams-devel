\documentclass[a4paper, 11pt]{article}
\usepackage[T1]{fontenc}
%\usepackage[utf8]{inputenc}
\usepackage[latin1]{inputenc}
%\usepackage[frenchb]{babel}
\usepackage{aeguill}
\usepackage{textcomp}
\usepackage[gen]{eurosym}
%\usepackage{natbib}
%\usepackage{theorem}

\usepackage{amsmath}
\usepackage{amsfonts}
\usepackage{amssymb}
\usepackage{amsthm}
\usepackage{algorithm,algorithmic}

%\usepackage[dvips]{graphicx}
\usepackage{fullpage}
\usepackage{graphics}
\usepackage{color}
\usepackage{epsfig}
\usepackage{floatflt}
\usepackage{tabularx}
\usepackage{verbatim}
\usepackage{url}
\usepackage{lmodern}
\usepackage{listings}
\usepackage{hevea}
%\usepackage{hyperref}

\definecolor{dkgreen}{rgb}{0.0,0.5,0.0}
\definecolor{gray}{rgb}{0.5,0.5,0.5}
\definecolor{mauve}{rgb}{1.0,0,0}
%\definecolor{mauve}{rgb}{0.58,0,0.82}
\definecolor{lightblue}{rgb}{0.9,0.93,1.0}

../../../../doc/macros.tex

%%\addtolength{\oddsidemargin}{-.5in}
%%\addtolength{\evensidemargin}{-.5in}
%%\addtolength{\textwidth}{1in}

%%\addtolength{\topmargin}{-.5in}
%%\addtolength{\textheight}{1in}


\ifdefined \lstavoidwhitepre
\lstavoidwhitepre
\lstdefinestyle{colors}{keywordstyle={\bf\color{blue}}, commentstyle={\color{dkgreen}}, stringstyle=\color{mauve}}
\lstset{language=Matlab, style=colors, backgroundcolor=\color{lightblue}}
\else
\lstdefinestyle{colors}{keywordstyle={\bf\color{blue}}, commentstyle={\color{dkgreen}}, stringstyle=\color{mauve}}
\lstset{language=Matlab, style=colors,basicstyle=\footnotesize, backgroundcolor=\color{lightblue}, showspaces=false, showstringspaces=false, showtabs=false, frame=single, breaklines=true, basewidth=5.5pt }
\fi

\title{SPAMS dictionary learning applied to text documents}
\date{\today}

\author{Jean-Paul Chi�ze}


\begin{document}
\maketitle

\tableofcontents
\newpage

\section{Introduction}
The goal is to learn a sparse dictionary from a set of text documents and then to compute a sparse decomposition 
of other documents on this dictionary.
The user chooses the size of the dictionary and can impose a structure on it.\\
The program uses the dictionary learning algorithms from SPAMS (SPArse Modeling Software by Julien Mairal).
It is written in python and offers a graphical user interface and a visual
represenation of the dictionary and decompositions.\\

Given a vocabulary of $N$ words, 
a document is represented by a  vector $x$ of $N$  elements, where $x_j$ is the probability
of occurence of word $j$ in the document ($\sum_{j=1}^N x_j =1$).

$X$ being a set of $M$ documents vectors $x^i$, the dictionary $D$ is a $N$x$K$ matrix, so that
\begin{equation}
\label{eq1}
  \min_{\D \in \C} \lim_{n \to +\infty} \frac{1}{n} \sum_{i=1}^n \min_{\alphab^i} \Big( \frac{1}{2} \|\x^i-\D\alphab^i\|_2^2 + \psi(\alphab^i)\Big).
\end{equation}
$\psi$ is a sparsity-inducing regularizer and $\C$ is a constraint set for the dictionary.
You will find a detailed description in the documentation of SPAMS.

\section{Quickstart}
You need python (2.6 or 2.7) with packages pyQT4, python-opengl, pyQGLViewer.
This program has been developped on linux.
As far as I know, on windows, it is necessary to build the pyQGLViewer module from source code, 
which  could be painful.

\begin{itemize}
\item Install package spams-python.
\item extract contents of text\_dictlearn tar file.
\item execute {\it python tex\_dictlearn/qtext-dict.py}
\end{itemize}
The program comes with some example files in subdirectory {\it example}.
You can use them for a quick trial.
\begin{itemize}
\item {\it example/vocab.txt} a vocabulary made of the 96 french {\it ``departements''}.
\item {\it example/documents.txt} 100 randomly generated documents
\item {\it example/mygraph} a example of graph structure that you can use for dictionary learning
\item {\it example/mytree} a example of tree structure, which has been used for {\it test.dmp}.
\item {\it example/test.dmp} preprocessed data that can be loaded with menu  entry {\bf File/Load}. It contains a dictionary matrix and a decomposition matrix that can be directly viewed. The data was generated with {\it vocab.txt, documents.txt,
mytree.txt}.
\end{itemize}

\subsection{Running examples :}
\begin{itemize}
\item {\bf pre-computed data} :
  \begin{enumerate}
    \item select menu {\bf File/Load} and open {\it example/test.dmp}.
    \item select menu {\bf File/View dictionary} and play with the Dictionary
      display window.
    \item select menu {\bf File/View decomposition} and play with the Decomposition
      display window.
  \end{enumerate}
\item {\bf vocabulary and documents test files} :
  \begin{enumerate}
    \item click on  {\bf Vocabulary} button and select {\it example/vocab.txt}
    \item click on  {\bf Texts for training} button and select {\it example/documents.txt}
    \item click on {\bf tree struct} of {\bf choose dictionary structure} option button, 
      and select file {\it example/mytree}
    \item select menu {\bf File/Learn dictionary} and wait until ``END'' is written 
      in the console window.
    \item select menu {\bf File/Compute decomposition} and wait until ``END DECOMP''
    \item select menu {\bf File/View dictionary} and play with the Dictionary
      display window.
    \item select menu {\bf File/View decomposition} and play with the Decomposition
      display window.
  \end{enumerate}
\end{itemize}
\section{Data description}
The vocabulary must be given as a $N$ lines text file (1 word per line).\\
Each document is a set of $N$ ascii float numbers representing the occurence of words. Their sum must be equal to 1.
Several documents can be gathered in the same file.\\
The variables of the decomposition are the columns of the dictionary. Variables can be :
\begin{itemize}
\item[-] unstructured : just specify the number $K$ of variables.
\item[-]  structured as a symetric tree : specify the number of children at each level.
\item[-] structured as a graph or tree :  see descriptions  of {\bf spams.fistaGraph} and {\bf spams.fistaTree} for the properties of these 
structures. The programe expects a text file of $p$ lines, where each line describes a node of the graph or tree.
Each line must be of the form \\
{\em node-num weight [list-of-variables] -> list-of-children }
\begin{itemize}
\item {\em node-num}\,: $0 .. (p-1)$
\item {\em weight}\,: float value, 1. if omitted
\item {\em list-of-variables}\,: list  of space separated integers representing the variables (i.e dictionary columns) indices attached to the node. May be empty, but [] are mandatory. 
\item {\em list-of-children} : list of space separated  node nums. May  be empty ('->' can be omitted in this case)
\end{itemize}
In the graph case, the same variable may appear in several nodes.
\end{itemize}
\section{User interface}
\subsection{Main window}

\begin{itemize}
\item menu {\bf file} : 
Some menu items are active only if the data they need is available.
\begin{itemize}
\item {\bf Learn dictionary} : read data files (vocabulary, training texts, graph or tree file), and run the dictionary learning program.
\item {\bf Compute decomposition} : run the decomposition program. Is active only if leaning has been successfull.
\item {\bf View dictionary} : open the dictionary visualization window.
\item {\bf View decomposition} : open a window to visualize the decomposition of a document.
\item {\bf Hide console} : hide the console. The item will be renamed {\bf Show~console}.
\item {\bf dump} : dump available data to a file (dictionary matrix, params, etc.).
May be partial (for example without decomposition matrix if not yet computed).
\item {\bf load} : load a file of previously dumped data.
\end{itemize}
\item menu {\bf Options} : has only item {\bf dictionary params}.
Opens the options window (see below).
\item button {\bf Vocabulary} : select the file containing the vocabulary.
  This button is displayed in red until a vocabulary is selected, and afterwards in green.
\item button {\bf Texts for training} : select the files defining texts to use for dictionary learning. It is disabled until a dictionary is selected.
If it is red, you must select document files to be able to run dictionary learning.
If it is green you can replace current training set.
This button will turn red each time you select a new vocabulary.
\item button {\bf Texts to decompose} : select the files defining the texts to
  decompose. The list is initialized with the ``Texts~for~training''.
This button is disabled until you select a vocabulary.
\item menu {\bf choose dictionary structure} : select kind of dictionnary structure.
\begin{itemize}
\item {\bf unstructured} : opens a dialog window to set the number $K$ of variables.
\item {\bf graph struct} : select a file defining a {\it graph} structure.
\item {\bf tree struct} : select a file defining a {\it tree} structure.
\item {\bf symetric tree} : enter a coma separated list of integers representing the number
of children at each level of the tree (for ex 2,2,2).
\end{itemize}
\item {\bf Console} : displays messages from the application.
  Errors are displayed in red. This window can be detached from the main window.
\end{itemize}
\subsection{Options window}
This is the window where you can set the parameters for 
{\bf spams.structTrainDL} (see description in documentation of SPAMS).\\
The current mode, as set in the {\bf choose~dictionary~structure} menu of main window,
is displayed at the top line. Each mode ("Unstructured", "Graph", "Tree") has its own
parameters that can be set in a separate tab.
\begin{itemize}
\item button {\bf Reset} : reset parameters to default values.
\item tabs for mode : they define {\it lambda, gamma1} and the {\it regularization~type}.
Default values and possible choices for regularization depend on the mode.
\item button {\bf Advanced} : show/hide some {\it advanced} parameters, common to all modes.
\end{itemize}
\subsection{Dictionary visualization}
This window offers a 3D representation of the dictionary structure.
There are 2 main modes of display :
\begin{itemize}
\item by nodes (menu item {\bf Show~nodes}) : 
Ecah node of the dictionary is represented as a red cube with the list of variables it contains.
For unstructured dictionaries, nodes are independant and variable $i$ is associated to node $i$.
Cubes can have 3 sizes depending on the number of associated variables : small for
no variables, medium for 1 variable, and large for more than 1 variable.
\item by variables (menu items {\bf Show~variables} and {\bf Dict~columns}) : each variable is represented by a red cube with the variable index.\\
If menu item {\bf Dict~columns} is selected, the vocabulary is listed on the left,
and the values of components of each column are represented by green rectangles.
\end{itemize}
In the {\bf Show~nodes} or {\bf Show~variables} mode, you can display the 10 most significant words of variables : 
\begin{enumerate}
\item : A click with Shift+left\_mouse\_button on a cube will toggle display of
most important words for each underlying variable.
\item : button {\bf Show/Hide all words} will show or hide words for all variables.
\end{enumerate}
Selected cubes are displayed in green.
\subsection{Decomposition visualization}
This window displays the decomposition of a document on the current dictionary,
that is the vector ${\alpha}^i$ of equation \eqref{eq1}.
The document to display can be choosen by its index ({\bf num~of~text~to~view}).
It is also possible to increment/decrement the document index by hitting right/left arrows in the graphic window.
Three display modes are available :
\begin{itemize}
\item menu item {\bf Decomposition on nodes} : nodes of dictionary are drawn as cubes with
a size proportional to the sum of values of $\alpha^i_j$ for variables j attached to the node.
There are 6 possible sizes, in particular 0 if $\alpha^i_j$ is zero.
Nodes where size is maximum are drawn in green and the 10 most significant words of each 
of their variables is displayed.
\item menu item {\bf Decomposition on variables} : variables are drawn as cubes with
a size proportional to $\alpha^i_j$.
As above, most significant words are displayed beside variables with maximum coefficient.
\item menu item {\bf Dict columns} : as above, variables are represented by cubes of size
proportional to $\alpha^i_j$, but no words are displayed.
Instead, the variables vectors (i.e dictionary columns) are drawn as barcharts, as in
the dictionary window.
\end{itemize}
\end{document}
